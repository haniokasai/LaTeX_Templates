\providecommand{\main}{..}
\documentclass[dvipdfmx]{jsarticle}
\usepackage[dvipdfmx]{graphicx}
\usepackage{subfiles}

\usepackage{ascmac}
\usepackage{lib_ulinej}
\usepackage{lib_bendarrow}
\usepackage{amsmath}
\usepackage{datetime}
\usepackage{comment}
\usepackage{framed}
\usepackage{arydshln}
\usepackage{breqn}
\usepackage{graphicx}
\graphicspath{{\main/imgs/}{imgs/}}

%https://stackoverflow.com/questions/722613/centering-a-table-wider-than-the-text-column
\usepackage{chngpage} % allows for temporary adjustment of side margins

%スクリプトの記述用
\usepackage{color}
\usepackage{listings,lib_jlisting}

%適切な場所に画像等を表示
\usepackage{here}

%listoftables用
\usepackage{tocloft}

%余白
\usepackage[top=4cm, bottom=3cm, left=2cm, right=2cm]{geometry}

%bibtex用
\usepackage{url}
\usepackage{natbib}

%platex文字化け対策
\usepackage[utf8]{inputenc}

%タイトルのスペース
\usepackage{titlesec}

%表のキャプションをサイト
\usepackage[hidelinks,colorlinks=false]{hyperref}
\usepackage{lib_pxjahyper}%日本語の文字化け防止

%ページずれ防止


%言語の設定
\lstset{%
  tabsize=4, % Tabを何文字幅にするかの指定
%  language={R}, % 言語の指定
  basicstyle={\ttfamily\small}, % 標準の書体の指定
  backgroundcolor={\color[gray]{.90}},%背景色
%  identifierstyle={¥scriptsize}, % キーワードでない文字の書体
%  commentstyle={¥scriptsize¥itshape},% コメントの書体
%  keywordstyle={¥scriptsize¥bfseries},% キーワードの書体
%  ndkeywordstyle={¥scriptsize},% キーワードの書体その2
%  stringstyle={¥scriptsize¥ttfamily}, % ""で囲まれた文字の書体
  columns=[l]{fullflexible},% 書体による列幅の違いを調整するか
%  frame={tbl}, %枠追加
%  frameround=ftff,%フレーム角の形状
  framesep=5pt,%本文からframeまでの間隔
  breaklines=true, % 行が長くなった場合の改行
  numbers=left, % 左側に行番号追加
  xrightmargin=0zw,%
  xleftmargin=1zw,%
  numberstyle={\scriptsize},% 行番号のスタイル
  stepnumber=1, % 行番号をいくつ飛ばしで表示するか
%  numbersep=1zw,% 行番号と本文の間隔。デフォルトは10pt
  lineskip=-0.5ex, % 行間の調整
  showstringspaces=false
}

\title{\vspace{-3cm} タイトル} 
\author{\vspace{-1cm} 筆者名\thanks{所属とメアド}}
\date{最終更新:~\today~\currenttime~(JST)}

\begin{document}

    \maketitle

    \setcounter{tocdepth}{3}
    % 3 = \subsubsectionまで
    \tableofcontents

    \makeatletter
    \renewcommand\listoftables{%
        \@starttoc{lot}%
    }

    
    \clearpage
    \subfile{pages/1_イントロダクション}

  %  \newpage
    \phantomsection
        
    \subfile{pages/5_付録}

    \newpage
    \phantomsection
    \section*{参考文献}
    \addcontentsline{toc}{section}{参考文献} 
    \bibliographystyle{jecon}
    %タイトルを消す
    \renewcommand{\bibsection}{}
    \bibliography{kncitations}


\end{document}



%下 記述例
%\begin{itembox}[c]{ヒントと注意}
%   まずは UMP を解かないと話が始まりません。
%\end{itembox}
%
%  \includegraphics[width=16cm]{toi42.png}
%
%\text{rtarrow}\rtarrow,\text{rbarrow}\rbarrow,\text{ltarrow}\ltarrow,\text{lbarrow}\lbarrow \\\\
%
%   \begin{enumerate}
%    \setlength{\leftskip}{2.5mm}
%    \renewcommand{\labelenumi}{Step \theenumi.}
%    \item A
%    \item B
%   \end{enumerate}
%
%   \begin{screen}
%       (a)~(c)
%       \begin{enumerate}
%           \renewcommand{\labelenumi}{(\alph{enumi})}
%           \item A
%           \item B
%       \end{enumerate}
%   \end{screen}
%
%\begin{shadebox}
%    ①~
%    \begin{enumerate}
%        \renewcommand{\labelenumi}{\textcircled{\scriptsize \theenumi}}
%        \item プレーヤーはAとBの2人
%        \item 同時に意志決定
%    \end{enumerate}
%\end{shadebox}
%
%\begin{table}[h]
%      \begin{center}
%        \begin{tabular}{|c|c|c|c|} \hline
%            A\B&グー&チョキ&パー \\  \hline
%            グー&0, 0&3, 0&0, 6 \\ \hline
%        \end{tabular}
%      \end{center}
%      \caption{(1)}
%      \label{(1)}
%\end{table}
%
%\begin{eqnarray*}
%    m &=& \frac{5}{21} \\
%    n &=& \frac{10}{21}
%\end{eqnarray*}
%
%増減表をつくる(必要な部分だけ埋めた)。
%\begin{table}[htb]
%  \centering
%  \begin{tabular}{c||c|c|c|c|c|c|c|c|c}
%    $x$               & 0    &          & &     & 10 \\ \hline
%    $u'$              &  &       &           & &    \\
%    $u^{\prime\prime}$&  &       &        &  & \\ \hline
%    $u$               &  &  & $\mbox{極大値}\atop    $ &   &  
%  \end{tabular}
%\end{table}
